\let\negmedspace\undefined
\let\negthickspace\undefined
%\RequirePackage{amsmath}
\documentclass[journal,12pt,twocolumn]{IEEEtran}
%
% \usepackage{setspace}
 \usepackage{gensymb}
%\doublespacing
 \usepackage{polynom}
%\singlespacing
%\usepackage{silence}
%Disable all warnings issued by latex starting with "You have..."
%\usepackage{graphicx}
\usepackage{amssymb}
%\usepackage{relsize}
\usepackage[cmex10]{amsmath}
%\usepackage{amsthm}
%\interdisplaylinepenalty=2500
%\savesymbol{iint}
%\usepackage{txfonts}
%\restoresymbol{TXF}{iint}
%\usepackage{wasysym}
\usepackage{amsthm}
%\usepackage{pifont}
%\usepackage{iithtlc}
% \usepackage{mathrsfs}
% \usepackage{txfonts}
 \usepackage{stfloats}
% \usepackage{steinmetz}
 \usepackage{bm}
% \usepackage{cite}
% \usepackage{cases}
% \usepackage{subfig}
%\usepackage{xtab}
\usepackage{longtable}
%\usepackage{multirow}
%\usepackage{algorithm}
%\usepackage{algpseudocode}
\usepackage{enumitem}
\usepackage{mathtools}
\usepackage{tikz}
% \usepackage{circuitikz}
% \usepackage{verbatim}
%\usepackage{tfrupee}
\usepackage[breaklinks=true]{hyperref}
%\usepackage{stmaryrd}
%\usepackage{tkz-euclide} % loads  TikZ and tkz-base
%\usetkzobj{all}
\usepackage{listings}
    \usepackage{color}                                            %%
    \usepackage{array}                                            %%
    \usepackage{longtable}                                        %%
    \usepackage{calc}                                             %%
    \usepackage{multirow}                                         %%
    \usepackage{hhline}                                           %%
    \usepackage{ifthen}                                           %%
  %optionally (for landscape tables embedded in another document): %%
    \usepackage{lscape}     
% \usepackage{multicol}
% \usepackage{chngcntr}
%\usepackage{enumerate}
\usepackage{tfrupee}

%\usepackage{wasysym}
%\newcounter{MYtempeqncnt}
\DeclareMathOperator*{\Res}{Res}
\DeclareMathOperator*{\equals}{=}
%\renewcommand{\baselinestretch}{2}
%\renewcommand\thesection{\arabic{section}}
%\renewcommand\thesubsection{\thesection.\arabic{subsection}}
%\renewcommand\thesubsubsection{\thesubsection.\arabic{subsubsection}}

%\renewcommand\thesectiondis{\arabic{section}}
%\renewcommand\thesubsectiondis{\thesectiondis.\arabic{subsection}}
%\renewcommand\thesubsubsectiondis{\thesubsectiondis.\arabic{subsubsection}}

% correct bad hyphenation here
\hyphenation{op-tical net-works semi-conduc-tor}
\def\inputGnumericTable{}                                 %%

\lstset{
%language=C,
frame=single, 
breaklines=true,
columns=fullflexible
}
%\lstset{
%language=tex,
%frame=single, 
%breaklines=true
%}
\begin{document}

%

\newtheorem{theorem}{Theorem}[section]
\newtheorem{problem}{Problem}
\newtheorem{proposition}{Proposition}[section]
\newtheorem{lemma}{Lemma}[section]
\newtheorem{corollary}[theorem]{Corollary}
\newtheorem{example}{Example}[section]
\newtheorem{definition}[problem]{Definition}
%\newtheorem{thm}{Theorem}[section] 
%\newtheorem{defn}[thm]{Definition}
%\newtheorem{algorithm}{Algorithm}[section]
%\newtheorem{cor}{Corollary}
\newcommand{\BEQA}{\begin{eqnarray}}
        \newcommand{\EEQA}{\end{eqnarray}}
\newcommand{\define}{\stackrel{\triangle}{=}}
\newcommand*\circled[1]{\tikz[baseline=(char.base)]{
        \node[shape=circle,draw,inner sep=2pt] (char) {#1};}}
\bibliographystyle{IEEEtran}
%\bibliographystyle{ieeetr}
\providecommand{\mbf}{\mathbf}
\providecommand{\pr}[1]{\ensuremath{\Pr\left(#1\right)}}
\providecommand{\qfunc}[1]{\ensuremath{Q\left(#1\right)}}
\providecommand{\sbrak}[1]{\ensuremath{{}\left[#1\right]}}
\providecommand{\lsbrak}[1]{\ensuremath{{}\left[#1\right.}}
\providecommand{\rsbrak}[1]{\ensuremath{{}\left.#1\right]}}
\providecommand{\brak}[1]{\ensuremath{\left(#1\right)}}
\providecommand{\lbrak}[1]{\ensuremath{\left(#1\right.}}
\providecommand{\rbrak}[1]{\ensuremath{\left.#1\right)}}
\providecommand{\cbrak}[1]{\ensuremath{\left\{#1\right\}}}
\providecommand{\lcbrak}[1]{\ensuremath{\left\{#1\right.}}
\providecommand{\rcbrak}[1]{\ensuremath{\left.#1\right\}}}
\theoremstyle{remark}
\newtheorem{rem}{Remark}
\newcommand{\sgn}{\mathop{\mathrm{sgn}}}
\providecommand{\fourier}{\overset{\mathcal{F}}{ \rightleftharpoons}}
%\providecommand{\hilbert}{\overset{\mathcal{H}}{ \rightleftharpoons}}
\providecommand{\system}{\overset{\mathcal{H}}{ \longleftrightarrow}}
%\newcommand{\solution}[2]{\textbf{Solution:}{#1}}
\newcommand{\solution}{\noindent \textbf{Solution: }}
\newcommand{\cosec}{\,\text{cosec}\,}
\providecommand{\dec}[2]{\ensuremath{\overset{#1}{\underset{#2}{\gtrless}}}}
\newcommand{\myvec}[1]{\ensuremath{\begin{pmatrix}#1\end{pmatrix}}}
\newcommand{\mydet}[1]{\ensuremath{\begin{vmatrix}#1\end{vmatrix}}}
\newcommand*{\permcomb}[4][0mu]{{{}^{#3}\mkern#1#2_{#4}}}
\newcommand*{\perm}[1][-3mu]{\permcomb[#1]{P}}
\newcommand*{\comb}[1][-1mu]{\permcomb[#1]{C}}
%\numberwithin{equation}{section}
%\numberwithin{figure}{section}
%\numberwithin{table}{section}
%\numberwithin{equation}{subsection}
%\numberwithin{problem}{section}
%\numberwithin{definition}{section}
\makeatletter
\@addtoreset{figure}{problem}
\makeatother
\let\StandardTheFigure\thefigure
\let\vec\mathbf
%\renewcommand{\thefigure}{\theproblem.\arabic{figure}}
%\renewcommand{\thefigure}{\theproblem}
%\setlist[enumerate,1]{before=\renewcommand\theequation{\theenumi.\arabic{equation}}
%\counterwithin{equation}{enumi}
%\renewcommand{\theequation}{\arabic{subsection}.\arabic{equation}}
\def\putbox#1#2#3{\makebox[0in][l]{\makebox[#1][l]{}\raisebox{\baselineskip}[0in][0in]{\raisebox{#2}[0in][0in]{#3}}}}
\def\rightbox#1{\makebox[0in][r]{#1}}
\def\centbox#1{\makebox[0in]{#1}}
\def\topbox#1{\raisebox{-\baselineskip}[0in][0in]{#1}}
\def\midbox#1{\raisebox{-0.5\baselineskip}[0in][0in]{#1}}
\title{
    %\logo{
    %Computational Approach to School Geometry
    Random Numbers
    %	}
}
\author{ Varun Gupta (cs21btech11060)% <-this % stops a space
}
\graphicspath{{figures/}}
%\title{
%	\logo{Matrix Analysis through Octave}{\begin{center}\includegraphics[scale=.24]{tlc}\end{center}}{}{HAMDSP}
%}
% paper title
% can use linebreaks \\ within to get better formatting as desired
%\title{Matrix Analysis through Octave}
%
%
% author names and IEEE memberships
% note positions of commas and nonbreaking spaces ( ~ ) LaTeX will not break
% a structure at a ~ so this keeps an author's name from being broken across
% two lines.
% use \thanks{} to gain access to the first footnote area
% a separate \thanks must be used for each paragraph as LaTeX2e's \thanks
% was not built to handle multiple paragraphs
%
%\author{<-this % stops a space
%\thanks{}}
%}
% note the % following the last \IEEEmembership and also \thanks - 
% these prevent an unwanted space from occurring between the last author name
% and the end of the author line. i.e., if you had this:
% 
% \author{....lastname \thanks{...} \thanks{...} }
%                     ^------------^------------^----Do not want these spaces!
%
% a space would be appended to the last name and could cause every name on that
% line to be shifted left slightly. This is one of those "LaTeX things". For
% instance, "\textbf{A} \textbf{B}" will typeset as "A B" not "AB". To get
% "AB" then you have to do: "\textbf{A}\textbf{B}"
% \thanks is no different in this regard, so shield the last } of each \thanks
% that ends a line with a % and do not let a space in before the next \thanks.
% Spaces after \IEEEmembership other than the last one are OK (and needed) as
% you are supposed to have spaces between the names. For what it is worth,
% this is a minor point as most people would not even notice if the said evil
% space somehow managed to creep in.
%\WarningFilter{latex}{LaTeX Warning: You have requested, on input line 117, version}
% The paper headers
%\markboth{Journal of \LaTeX\ Class Files,~Vol.~6, No.~1, January~2007}%
%{Shell \MakeLowercase{\textit{et al.}}: Bare Demo of IEEEtran.cls for Journals}
% The only time the second header will appear is for the odd numbered pages
% after the title page when using the twoside option.
% 
% *** Note that you probably will NOT want to include the author's ***
% *** name in the headers of peer review papers.                   ***
% You can use \ifCLASSOPTIONpeerreview for conditional compilation here if
% you desire.
% If you want to put a publisher's ID mark on the page you can do it like
% this:
%\IEEEpubid{0000--0000/00\$00.00~\copyright~2007 IEEE}
% Remember, if you use this you must call \IEEEpubidadjcol in the second
% column for its text to clear the IEEEpubid mark.
% make the title area
\maketitle
%\renewcommand{\theequation}{\theenumi}
%\begin{abstract}
%%\boldmath
%In this letter, an algorithm for evaluating the exact analytical bit error rate  (BER)  for the piecewise linear (PL) combiner for  multiple relays is presented. Previous results were available only for upto three relays. The algorithm is unique in the sense that  the actual mathematical expressions, that are prohibitively large, need not be explicitly obtained. The diversity gain due to multiple relays is shown through plots of the analytical BER, well supported by simulations. 
%
%\end{abstract}
% IEEEtran.cls defaults to using nonbold math in the Abstract.
% This preserves the distinction between vectors and scalars. However,
% if the journal you are submitting to favors bold math in the abstract,
% then you can use LaTeX's standard command \boldmath at the very start
% of the abstract to achieve this. Many IEEE journals frown on math
% in the abstract anyway.
% Note that keywords are not normally used for peerreview papers.
%\begin{IEEEkeywords}
%Cooperative diversity, decode and forward, piecewise linear
%\end{IEEEkeywords}
% For peer review papers, you can put extra information on the cover
% page as needed:
% \ifCLASSOPTIONpeerreview
% \begin{center} \bfseries EDICS Category: 3-BBND \end{center}
% \fi
%
% For peerreview papers, this IEEEtran command inserts a page break and
% creates the second title. It will be ignored for other modes.
%\IEEEpeerreviewmaketitle
\begin{abstract}
    This document contains the solution for Random Numbers
\end{abstract}

%template ends here

%main text begins
\textbf{Problem 1.1:} Generate $10^6$ samples of $U$ using a C program and save into a file called uni.dat.
\solution
\begin{lstlisting}
    wget https://github.com/procodervarun/RandomNumbers/blob/master/codes/exrand.c
    wget https://github.com/procodervarun/RandomNumbers/blob/master/codes/coeffs.h
    gcc exrand.c
    ./a.out
\end{lstlisting}
\textbf{Problem 1.2:} Load the uni.dat file into python and plot the empirical CDF of $U$ using the samples in uni.dat. The CDF is defined as\\
\solution
\begin{figure}[!ht]
\includegraphics[width=\columnwidth]{./figs/uni_cdf}
\caption{The CDF of $U$}
\label{fig:uni_cdf}
\end{figure}\\
\textbf{Problem 1.3:} Find a theoretical solution for $F_U(x)$\\
\solution 
\begin{align}
    F_U(x) = \pr{U \leq x} = \int_{-\infty}^{x}p_U(u)du
\end{align}
\begin{enumerate}
    \item Case 1: $x < 0$: $p_X(x) = 0 \Rightarrow F_U(x) = 0$
    \item Case 2: $0 \leq x < 1$: $F_U(x) = \int_{0}^{x}du = x$
    \item Case 3: $x \geq 1$: Put x = 1 to get $F_U(x) = 1$ 
\end{enumerate}
\begin{align}
    F_U(x) = 
    \begin{cases}
        0 & x < 0 \\
        x & 0 \leq x < 1 \\
        1 & x \geq 1 \\
    \end{cases}
\end{align}
\textbf{Problem 1.4: }The mean of $U$ is defined as
%
\begin{equation}
E\sbrak{U} = \frac{1}{N}\sum_{i=1}^{N}U_i
\end{equation}
%
and its variance as
%
\begin{equation}
\text{var}\sbrak{U} = E\sbrak{U- E\sbrak{U}}^2 
\end{equation}

Write a C program to  find the mean and variance of $U$.\\
\solution
\begin{lstlisting}
    wget https://github.com/procodervarun/RandomNumbers/blob/master/codes/uni.c
    gcc uni.c
    ./a.out
\end{lstlisting}
\textbf{Problem 1.5:} Verify your result theoretically given that $E[U^k] = \int_{-\infty}^{\infty}x^k dF_U(x)$\\
\solution
\begin{align}
    E[U] = \int_{0}^{1}xdx &= \frac{1}{2}\\
\end{align}
$\Rightarrow$ theoretically empirical mean is 0.50 which closely agrees with numerical value of 0.500007\\
\begin{align}
    E[U^2] = \int_{0}^{1}x^2dx &= \frac{1}{3}\\
    \text{var}\sbrak{U} &= E\sbrak{U-E\sbrak{U}}^2\\
    &= E[U^2-2UE\sbrak{U}+E^2[U]]\\ 
    &= E[U^2]-E^2[U] = \frac{1}{12}
\end{align}
$\Rightarrow$ theoretically empirical variance is 0.083333 which closely agrees with numerical value of 0.083301\\
\textbf{Problem 2.1: }Generate $10^6$ samples of the random variable
\begin{equation}
X = \sum_{i=1}^{12}U_i -6
\end{equation}
using a C program, where $U_i, i = 1,2,\dots, 12$ are  a set of independent uniform random variables between 0 and 1 and save in a file called gau.dat\\
\solution
\begin{lstlisting}
    wget https://github.com/procodervarun/RandomNumbers/blob/master/codes/exrand.c
    wget https://github.com/procodervarun/RandomNumbers/blob/master/codes/coeffs.h
    gcc exrand.c
    ./a.out
\end{lstlisting}
\textbf{Problem 2.2: }Load gau.dat in python and plot the empirical CDF of $X$ using the samples in gau.dat. What properties does a CDF have?\\
\solution
\begin{figure}[!ht]
    \centering
    \includegraphics[width=\columnwidth]{./figs/gauss_cdf}
    \caption{The CDF of $X$}
    \label{fig:gauss_cdf}
\end{figure}\\
\textbf{Problem 2.3: }Load gau.dat in python and plot the empirical PDF of $X$ using the samples in gau.dat. The PDF of $X$ is defined as
\begin{align}
p_{X}(x) = \frac{d}{dx}F_{X}(x)
\end{align}
What properties does the PDF have?\\
\solution \\\\\\
\begin{figure}[!ht]
    \centering
    \includegraphics[width=\columnwidth]{./figs/gauss_pdf}
    \caption{The PDF of $X$}
    \label{fig:gauss_pdf}
\end{figure}\\
\textbf{Problem 2.4: }Find the mean and variance of $X$ by writing a C program.\\
\solution
\begin{lstlisting}
    wget https://github.com/procodervarun/RandomNumbers/blob/master/codes/gau.c
    wget https://github.com/procodervarun/RandomNumbers/blob/master/codes/coeffs.h
    gcc gau.c
    ./a.out
\end{lstlisting}
\textbf{Problem 2.5:} Given that $p_X(x)=\frac{1}{\sqrt{2\pi}}\exp^{-\frac{x^2}{2}}$ \& repeat the above exercise theoretically.\\
\solution
\begin{align}
    E[X] = \int_{-\infty}^{\infty}x\frac{1}{\sqrt{2\pi}}e^{-\frac{x^2}{2}}dx = 0
\end{align}
$\Rightarrow$ theoretically empirical mean is 0 which closely agrees with numerical value of 0.000326\\
\begin{align}
    var[X] &= E[X^2] - E^2[X]\\
    &= \int_{-\infty}^{\infty}x^2\frac{1}{\sqrt{2\pi}}\exp^{-\frac{x^2}{2}}dx = 1\\
\end{align}
$\Rightarrow$ theoretically empirical variance is 1 which closely agrees with numerical value of 1.000906\\
\textbf{Problem 3.1: }Generate samples of 
%
\begin{equation}
V = -2\ln\brak{1-U}
\end{equation}
%
and plot its CDF.\\
\solution
\begin{figure}[!ht]
    \centering
    \includegraphics[width=\columnwidth]{./figs/exp_cdf}
    \caption{The PDF of $X$}
    \label{fig:exp_cdf}
\end{figure}\\
\textbf{Problem 3.2} Find the theoretical expression for $F_V(x)$\\
\solution
\begin{align}
    v = f(u) = -2\ln(1-U)\\
    u = f^{-1}(v) = 1-e^{-\frac{v}{2}}\\
    \Rightarrow F_V(v) = F_U(1-e^{-\frac{v}{2}})\\
    F_V(v) =
    \begin{cases}
        0 & v<0\\
        1-e^{-\frac{v}{2}} &v \geq 0
    \end{cases}
\end{align}
\end{document}